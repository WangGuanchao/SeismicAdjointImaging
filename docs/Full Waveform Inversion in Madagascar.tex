%%% Template originaly created by Karol Kozioł (mail@karol-koziol.net) and modified for ShareLaTeX use

\documentclass[a4paper,11pt]{article}

\usepackage[T1]{fontenc}
\usepackage[utf8]{inputenc}
\usepackage{graphicx}
\usepackage{xcolor}

\renewcommand\familydefault{\sfdefault}
\usepackage{tgheros}
\usepackage[defaultmono]{droidmono}

\usepackage{amsmath,amssymb,amsthm,textcomp}
\usepackage{enumerate}
\usepackage{multicol}
\usepackage{tikz}
\usepackage[colorlinks,linkcolor=blue]{hyperref}
\usepackage{geometry}
\geometry{total={210mm,297mm},
left=20mm,right=20mm,%
bindingoffset=0mm, top=20mm,bottom=20mm}

\linespread{1.3}

\newcommand{\linia}{\rule{\linewidth}{0.5pt}}

% custom theorems if needed
\newtheoremstyle{mytheor}
    {1ex}{1ex}{\normalfont}{0pt}{\scshape}{.}{1ex}
    {{\thmname{#1 }}{\thmnumber{#2}}{\thmnote{ (#3)}}}

\theoremstyle{mytheor}
\newtheorem{defi}{Definition}

% my own titles
\makeatletter
\renewcommand{\maketitle}{
\begin{center}
\vspace{2ex}
{\huge \textsc{\@title}}
\vspace{1ex}
\\
\linia\\
\@author \hfill \@date
\vspace{4ex}
\end{center}
}
\makeatother
%%%

% custom footers and headers
\usepackage{fancyhdr}
\pagestyle{fancy}
\lhead{}
\chead{}
\rhead{}
\cfoot{}
\rfoot{Page \thepage}
\renewcommand{\headrulewidth}{0pt}
\renewcommand{\footrulewidth}{0pt}
%

% code listing settings
\usepackage{listings}
\lstset{
    language=Python,
    basicstyle=\ttfamily\small,
    aboveskip={1.0\baselineskip},
    belowskip={1.0\baselineskip},
    columns=fixed,
    extendedchars=true,
    breaklines=true,
    tabsize=4,
    prebreak=\raisebox{0ex}[0ex][0ex]{\ensuremath{\hookleftarrow}},
    frame=lines,
    showtabs=false,
    showspaces=false,
    showstringspaces=false,
    keywordstyle=\color[rgb]{0.627,0.126,0.941},
    commentstyle=\color[rgb]{0.133,0.545,0.133},
    stringstyle=\color[rgb]{01,0,0},
    numbers=left,
    numberstyle=\small,
    stepnumber=1,
    numbersep=10pt,
    captionpos=t,
    escapeinside={<@}{@>}
}

%%%----------%%%----------%%%----------%%%----------%%%

\begin{document}

\title{Full Waveform Inversion}

\author{Hou, Sian - sianhou1987@outlook.com}

\date{Jan/09/2017}

\maketitle

\section*{Introduction}
This is an explantion of Full Waveform Inversion program in Madagascar (\url{https://github.com/ahay/src}) to help us understand the details of seismic inversion workflow. The author of code is Pengliang Yang and the theory can be found on \url{ http://www.reproducibility.org/RSF/book/xjtu/primer/paper_html/}. What's more, Karol Koziol published the {\LaTeX} template on ShareLatex \url{https://www.sharelatex.com/}.

Main points: 

1. Sub the source when recovery forward wavefield, see \hyperref[mainpoint_1]{line 309 in main( )}.

2. Calculate one time forward exploration for a better CG step, see \hyperref[mainpoint_2]{line 370-413 in main( )}

\section*{main( )}
main( ) in \url{$(RSFROOT)/src/user/pyang/Mfwi2d.c}.
\begin{lstlisting}[label={main},language=C,tabsize=4,caption=main()]

/* Time domain full waveform inversion 
   Note: This serial FWI is merely designed to help the understanding of 
   beginners. Enquist absorbing boundary condition (A2) is applied!
*/
/*
   Copyright (C) 2014  Xi'an Jiaotong University, UT Austin (Pengliang Yang)

   This program is free software; you can redistribute it and/or modify
   it under the terms of the GNU General Public License as published by
   the Free Software Foundation; either version 2 of the License, or
   (at your option) any later version.

   This program is distributed in the hope that it will be useful,
   but WITHOUT ANY WARRANTY; without even the implied warranty of
   MERCHANTABILITY or FITNESS FOR A PARTICULAR PURPOSE.  See the
   GNU General Public License for more details.

   You should have received a copy of the GNU General Public License
   along with this program; if not, write to the Free Software
   Foundation, Inc., 59 Temple Place, Suite 330, Boston, MA  02111-1307  USA

   Important references:
   [1] Clayton, Robert, and Bjaqrn Engquist. "Absorbing boundary 
	   conditions for acoustic and elastic wave equations." Bulletin 
	   of the Seismological Society of America 67.6 (1977): 1529-1540.
   [2] Tarantola, Albert. "Inversion of seismic reflection data in the 
       acoustic approximation." Geophysics 49.8 (1984): 1259-1266.
   [3] Pica, A., J. P. Diet, and A. Tarantola. "Nonlinear inversion 
       of seismic reflection data in a laterally invariant medium." 
       Geophysics 55.3 (1990): 284-292.
   [4] Dussaud, E., Symes, W. W., Williamson, P., Lemaistre, L., 
       Singer, P., Denel, B., & Cherrett, A. (2008). Computational 
       strategies for reverse-time migration. In SEG Technical Program 
       Expanded Abstracts 2008 (pp. 2267-2271).
   [5] Hager, William W., and Hongchao Zhang. "A survey of nonlinear
       conjugate gradient methods." Pacific journal of Optimization 
       2.1 (2006): 35-58.
*/

int main(int argc, char *argv[]) {
	
	//! variables on host 
	bool verb, precon, csdgather;
	int is, it, iter, niter, distx, distz, csd, rbell;
	int nz, nx, nt, ns, ng;

	//! parameters of acquisition geometery 
	int sxbeg, szbeg, gxbeg, gzbeg, jsx, jsz, jgx, jgz;
	float dx, dz, fm, dt, dtx, dtz, tmp, amp, obj1, obj, beta, epsil, alpha;
	float *dobs, *dcal, *derr, *wlt, *bndr, *trans, *objval;
	int *sxz, *gxz;		
	float **vv, **illum, **lap, **vtmp, **sp0, **sp1, **sp2, **gp0, **gp1, **gp2, **g0, **g1, **cg, *alpha1, *alpha2, **ptr=NULL;	

    //! time
	clock_t start, stop;
	
	//! I/O files
	sf_file vinit, shots, vupdates, grads, objs, illums;

	//! initialize Madagascar
	sf_init(argc,argv);

	//! set up I/O files 
	vinit=sf_input ("in");   	
	/* initial velocity model, unit=m/s */
	shots=sf_input("shots"); 	
	/* recorded shots from exact velocity model */
	vupdates=sf_output("out"); 	
	/* updated velocity in iterations */ 
	grads=sf_output("grads");  	
	/* gradient in iterations */ 
	illums=sf_output("illums");	
	/* source illumination in iterations */
	objs=sf_output("objs");		
	/* values of objective function in iterations */

	//! get parameters from velocity model and recorded shots 
	if (!sf_getbool("verb",&verb)) 		verb=true;			
	/* vebosity */
	if (!sf_histint(vinit,"n1",&nz)) 	sf_error("no n1");	
	/* nz */
	if (!sf_histint(vinit,"n2",&nx)) 	sf_error("no n2");	
	/* nx */
	if (!sf_histfloat(vinit,"d1",&dz)) 	sf_error("no d1");	
	/* dz */
	if (!sf_histfloat(vinit,"d2",&dx)) 	sf_error("no d2");	
	/* dx */
	if (!sf_getbool("precon",&precon)) 	precon=false;	
	/* precondition or not */
	if (!sf_getint("niter",&niter))   	niter=100;		
	/* number of iterations */
	if (!sf_getint("rbell",&rbell))	  	rbell=2;		
	/* radius of bell smooth */

	if (!sf_histint(shots,"n1",&nt)) 	sf_error("no nt");
	/* total modeling time steps */
	if (!sf_histint(shots,"n2",&ng)) 	sf_error("no ng");
	/* total receivers in each shot */
	if (!sf_histint(shots,"n3",&ns)) 	sf_error("no ns");
	/* number of shots */
	if (!sf_histfloat(shots,"d1",&dt)) 	sf_error("no dt");
	/* time sampling interval */
	if (!sf_histfloat(shots,"amp",&amp))sf_error("no amp");
	/* maximum amplitude of ricker */
	if (!sf_histfloat(shots,"fm",&fm)) 	sf_error("no fm");
	/* dominant freq of ricker */
	if (!sf_histint(shots,"sxbeg",&sxbeg)) 	sf_error("no sxbeg");
	/* x-begining index of sources, starting from 0 */
	if (!sf_histint(shots,"szbeg",&szbeg)) 	sf_error("no szbeg");
	/* x-begining index of sources, starting from 0 */
	if (!sf_histint(shots,"gxbeg",&gxbeg)) 	sf_error("no gxbeg");
	/* z-begining index of receivers, starting from 0 */
	if (!sf_histint(shots,"gzbeg",&gzbeg)) 	sf_error("no gzbeg");
	/* x-begining index of receivers, starting from 0 */
	if (!sf_histint(shots,"jsx",&jsx)) 	sf_error("no jsx");
	/* source x-axis  jump interval  */
	if (!sf_histint(shots,"jsz",&jsz)) 	sf_error("no jsz");
	/* source z-axis jump interval  */
	if (!sf_histint(shots,"jgx",&jgx)) 	sf_error("no jgx");
	/* receiver x-axis jump interval  */
	if (!sf_histint(shots,"jgz",&jgz)) 	sf_error("no jgz");
	/* receiver z-axis jump interval  */
	if (!sf_histint(shots,"csdgather",&csd)) 	
	sf_error("csdgather or not required");
	/* default, common shot-gather; if n, record at every point */

	//! set up I/O parameters 
	sf_putint(vupdates,"n1",nz);	
	sf_putint(vupdates,"n2",nx);
	sf_putint(vupdates,"n3",niter);
	sf_putfloat(vupdates,"d1",dz);
	sf_putfloat(vupdates,"d2",dx);
	sf_putint(vupdates,"d3",1);
	sf_putint(vupdates,"o3",1);
	sf_putstring(vupdates,"label1","Depth");
	sf_putstring(vupdates,"label2","Distance");
	sf_putstring(vupdates,"label3","Iteration");
	/* updated velocity in iterations */
	sf_putint(grads,"n1",nz);	
	sf_putint(grads,"n2",nx);
	sf_putint(grads,"n3",niter);
	sf_putfloat(grads,"d1",dz);
	sf_putfloat(grads,"d2",dx);
	sf_putint(grads,"d3",1);
	sf_putint(grads,"o3",1);
	sf_putstring(grads,"label1","Depth");
	sf_putstring(grads,"label2","Distance");
	sf_putstring(grads,"label3","Iteration");
	/* gradient in iterations */
	sf_putint(illums,"n1",nz);	
	sf_putint(illums,"n2",nx);
	sf_putint(illums,"n3",niter);
	sf_putfloat(illums,"d1",dz);
	sf_putfloat(illums,"d2",dx);
	sf_putint(illums,"d3",1);
	sf_putint(illums,"o3",1);
	/* source illumination in iterations */
	sf_putint(objs,"n1",niter);
	sf_putint(objs,"n2",1);
	sf_putfloat(objs,"d1",1);
	sf_putfloat(objs,"o1",1);
	/* values of objective function in iterations */
	dtx=dt/dx; 
	dtz=dt/dz; 
	csdgather=(csd>0)?true:false;

	//! allocate memory
	vv=sf_floatalloc2(nz, nx);
	/* updated velocity */
	vtmp=sf_floatalloc2(nz, nx);
	/* temporary velocity computed with epsil */
	sp0=sf_floatalloc2(nz, nx);
	/* source wavefield p0 */
	sp1=sf_floatalloc2(nz, nx);
	/* source wavefield p1 */
	sp2=sf_floatalloc2(nz, nx);
	/* source wavefield p2 */
	gp0=sf_floatalloc2(nz, nx);
	/* geophone/receiver wavefield p0 */
	gp1=sf_floatalloc2(nz, nx);
	/* geophone/receiver wavefield p1 */
	gp2=sf_floatalloc2(nz, nx);
	/* geophone/receiver wavefield p2 */
	g0=sf_floatalloc2(nz, nx);
	/* gradient at previous step */
	g1=sf_floatalloc2(nz, nx);
	/* gradient at curret step */
	cg=sf_floatalloc2(nz, nx);
	/* conjugate gradient */
	lap=sf_floatalloc2(nz, nx);
	/* laplace of the source wavefield */
	illum=sf_floatalloc2(nz, nx);
	/* illumination of the source wavefield */
	objval=(float*)malloc(niter*sizeof(float));
	/* objective/misfit function */
	wlt=(float*)malloc(nt*sizeof(float));
	/* ricker wavelet */
	sxz=(int*)malloc(ns*sizeof(int)); 
	/* source positions */
	gxz=(int*)malloc(ng*sizeof(int)); 
	/* geophone positions */
	bndr=(float*)malloc(nt*(2*nz+nx)*sizeof(float));
	/* boundaries for wavefield reconstruction */
	trans=(float*)malloc(ng*nt*sizeof(float));
	/* transposed one shot */
	dobs=(float*)malloc(ng*nt*sizeof(float));
	/* observed seismic data */
	dcal=(float*)malloc(ng*sizeof(float));	
	/* calculated/synthetic seismic data */
	derr=(float*)malloc(ns*ng*nt*sizeof(float));
	/* residual/error between synthetic and observation */
	alpha1=(float*)malloc(ng*sizeof(float));
	/* numerator of alpha, length=ng */
	alpha2=(float*)malloc(ng*sizeof(float));
	/* denominator of alpha, length=ng */

	//! initialize varibles
	sf_floatread(vv[0], nz*nx, vinit);
	memset(sp0[0], 0, nz*nx*sizeof(float));
	memset(sp1[0], 0, nz*nx*sizeof(float));
	memset(sp2[0], 0, nz*nx*sizeof(float));
	memset(gp0[0], 0, nz*nx*sizeof(float));
	memset(gp1[0], 0, nz*nx*sizeof(float));
	memset(gp2[0], 0, nz*nx*sizeof(float));
	memset(g0[0], 0, nz*nx*sizeof(float));
	memset(g1[0], 0, nz*nx*sizeof(float));
	memset(cg[0], 0, nz*nx*sizeof(float));
	memset(lap[0], 0, nz*nx*sizeof(float));
	memset(vtmp[0], 0, nz*nx*sizeof(float));
	memset(illum[0], 0, nz*nx*sizeof(float));
	/* set up zero for each array */
	
	for(it=0;it<nt;it++){
		tmp=SF_PI*fm*(it*dt-1.0/fm);
		tmp*=tmp;
		wlt[it]=(1.0-2.0*tmp)*expf(-tmp);
	}
	/* calculate source wavelet */
	
	if (!(sxbeg>=0 && szbeg>=0 && 
	      sxbeg+(ns-1)*jsx<nx && szbeg+(ns-1)*jsz<nz)) {
		sf_warning("sources exceeds the computing zone!\n"); 
		exit(1);
	}
	/* check source position */
	sg_init(sxz, szbeg, sxbeg, jsz, jsx, ns, nz); <@\label{sg_init_in_main}
	\hyperref[sg_init]{! GOTO sg\_init( )}@>
	/* shot position initialize */
	
	distx=sxbeg-gxbeg;
	distz=szbeg-gzbeg;
	if (csdgather){
		if(!(gxbeg>=0 && gzbeg>=0 && 
		      gxbeg+(ng-1)*jgx<nx && gzbeg+(ng-1)*jgz<nz &&
			  (sxbeg+(ns-1)*jsx)+(ng-1)*jgx-distx <nx  &&
			  (szbeg+(ns-1)*jsz)+(ng-1)*jgz-distz <nz)){ 
				  sf_warning("geophones exceeds the computing zone!\n"); 
				  exit(1); 
		}
	} else{
		if(!(gxbeg>=0 && gzbeg>=0 && 
			 gxbeg+(ng-1)*jgx<nx && gzbeg+(ng-1)*jgz<nz)){
			 	sf_warning("geophones exceeds the computing zone!\n"); 
			 	exit(1); 
		}
	}
	/* check receivers position */
	sg_init(gxz, gzbeg, gxbeg, jgz, jgx, ng, nz); <@\hyperref[sg_init]{! GOTO sg\_init( )}@>
	/* 
	 * receiver position initialize
	 * this code is available when csdgather==false 
	 */
	
	memset(bndr, 0, nt*(2*nz+nx)*sizeof(float));
	memset(dobs, 0, ng*nt*sizeof(float));
	memset(dcal, 0, ng*sizeof(float));
	memset(derr, 0, ns*ng*nt*sizeof(float));
	memset(alpha1, 0, ng*sizeof(float));
	memset(alpha2, 0, ng*sizeof(float));
	memset(dobs, 0, ng*nt*sizeof(float));	
	memset(objval, 0, niter*sizeof(float));
	/* set up zero for each array */
	
	for(iter=0; iter<niter; iter++){
		if(verb){
			start=clock();/* record starting time */
			sf_warning("iter=%d",iter);
		}
		sf_seek(shots, 0L, SEEK_SET);
		memcpy(g0[0], g1[0], nz*nx*sizeof(float));
		memset(g1[0], 0, nz*nx*sizeof(float));
		memset(illum[0], 0, nz*nx*sizeof(float));
		for(is=0;is<ns;is++){
			sf_floatread(trans, ng*nt, shots); 
			/* read shot gather */
			matrix_transpose(trans, dobs, nt, ng); <@\label{matrix_transpose_in_main}
			\hyperref[matrix_transpose]{! GOTO matrix\_transpose( )}@>
			/* transpose the matrix to get dobs */
			if(csdgather){
				gxbeg=sxbeg+is*jsx-distx;
				sg_init(gxz, gzbeg, gxbeg, jgz, jgx, ng, nz); <@\hyperref[sg_init]{! GOTO sg\_init( )}@>
			}
			/* receiver position initialize */
			
			memset(sp0[0], 0, nz*nx*sizeof(float));
			memset(sp1[0], 0, nz*nx*sizeof(float));
			for(it=0; it<nt; it++){
				add_source(sp1, &wlt[it], &sxz[is], 1, nz, true);
				<@\label{add_source_in_main}
				\hyperref[add_source]{! GOTO add\_source( )}@>
				/* add source */
				
				step_forward(sp0, sp1, sp2, vv, dtz, dtx, nz, nx);
				<@\label{step_forward_in_main}
				\hyperref[step_forward]{! GOTO step\_forward( )}@>
				/* forward exploration */
				
				ptr=sp0; sp0=sp1; sp1=sp2; sp2=ptr;
				/* update wavefield */
				
				rw_bndr(&bndr[it*(2*nz+nx)], sp0, nz, nx, true);
				<@\label{rw_bndr_in_main}
				\hyperref[rw_bndr]{! GOTO rw\_bndr( )}@>
				/* save boundary value for saving memory */

				record_seis(dcal, gxz, sp0, ng, nz);
				<@\label{record_seis_in_main}
				\hyperref[record_seis]{! GOTO record\_seis( )}@>
				/* save seismic record at recevier position */
				
				cal_residuals(dcal,&dobs[it*ng],&derr[is*ng*nt+it*ng],ng);
				<@\label{cal_residuals_in_main}
				\hyperref[cal_residuals]{! GOTO cal\_residuals( )}@>
				/* calculate record residual at recevier position */
			}			
			/* forward exploration complete */
			
			ptr=sp0; sp0=sp1; sp1=ptr;
			memset(gp0[0], 0, nz*nx*sizeof(float));
			memset(gp1[0], 0, nz*nx*sizeof(float));
			for(it=nt-1; it>-1; it--){
				rw_bndr(&bndr[it*(2*nz+nx)], sp1, nz, nx, false);
				<@\label{rw_bndr_in_main}
				\hyperref[rw_bndr]{! GOTO rw\_bndr( )}@>
				/* read boundary value for saving memory */
				
				step_backward(illum,lap,sp0,sp1,sp2,vv,dtz,dtx,nz,nx);
				<@\label{step_backward_in_main}
				\hyperref[step_backward]{! GOTO step\_backward( )}@>
				/* 
				 * this step is to recovery forward wavefield 
				 * via backward exploration and boundary condition
				 * illum is the source compensate 
				 * lap is the laplace operator * velocity^2
				 */
				
				add_source(sp1, &wlt[it], &sxz[is], 1, nz, false);
				<@\label{mainpoint_1}\hyperref[add_source]{! GOTO add\_source( )}@>
				/* sub source to elminate source in backward scattering */
				
				add_source(gp1, &derr[is*ng*nt+it*ng], gxz, ng, nz, true);
				<@\hyperref[add_source]{! GOTO add\_source( )}@>
				/* stack residual as backward scattering source */
				
				step_forward(gp0, gp1, gp2, vv, dtz, dtx, nz, nx);
				<@\hyperref[step_forward]{! GOTO step\_forward( )}@>
				/* backward scattering residual wavefield */

				cal_gradient(g1, lap, gp1, nz, nx);
				<@\label{cal_gradient_in_main}
				\hyperref[cal_gradient]{! GOTO cal\_gradient( )}@>
				/* calculate gradient via correlation */
				
				ptr=sp0; sp0=sp1; sp1=sp2; sp2=ptr;
				ptr=gp0; gp0=gp1; gp1=gp2; gp2=ptr;
				/* update wavefield */
			}
		}
		/* simulating complete */
	
		obj=cal_objective(derr, ng*nt*ns);
		<@\label{cal_objective_in_main}
		\hyperref[cal_objective]{! GOTO cal\_objective( )}@>
		/* obj = norm2(derr) */
		
		scale_gradient(g1, vv, illum, nz, nx, precon);
		<@\label{scale_gradient_in_main}
		\hyperref[scale_gradient]{! GOTO scale\_gradient( )}@>
		/* 
		 * g1 = 2.0*g1/velocity^2 
		 * IF precon == true DO source compensate 
		 */		
		sf_floatwrite(illum[0], nz*nx, illums);
		/* output illum */
		bell_smoothz(g1, illum, rbell, nz, nx);
		<@\label{bell_smoothz_in_main}
		\hyperref[bell_smoothz]{! GOTO bell\_smoothz( )}@>
		bell_smoothx(illum, g1, rbell, nz, nx);
		<@\label{bell_smoothx_in_main}
		\hyperref[bell_smoothx]{! GOTO bell\_smoothx( )}@>		
		/* smooth g1 while use illum as temp store */
		sf_floatwrite(g1[0], nz*nx, grads);
		/* output gradient */
		/* calculating gradient complete */
		
		if (iter>0) 
			beta=cal_beta(g0, g1, cg, nz, nx); 
		else 
			beta=0.0;
		<@\label{cal_beta_in_main}
		\hyperref[cal_beta]{! GOTO cal\_beta( )}@>
		/* calculate beta */
		cal_conjgrad(g1, cg, beta, nz, nx);
		<@\label{cal_conjgrad_in_main}
		\hyperref[cal_conjgrad]{! GOTO cal\_conjgrad( )}@>
		/* calculate cg direction */
		epsil=cal_epsilon(vv, cg, nz, nx);
		<@\label{cal_epsilon_in_main}
		\hyperref[cal_epsilon]{! GOTO cal\_epsilon( )}@>
		/* calculate cg step size */
		/* calculating CG direction complete */
		
		sf_seek(shots, 0L, SEEK_SET);
		memset(alpha1, 0, ng*sizeof(float));
		memset(alpha2, 0, ng*sizeof(float));
		cal_vtmp(vtmp, vv, cg, epsil, nz, nx);
		<@\label{cal_vtmp_in_main}\label{mainpoint_2}
		\hyperref[cal_vtmp]{! GOTO cal\_vtmp( )}@>
		/* update the velocity */
		for(is=0;is<ns;is++){
			sf_floatread(trans, ng*nt, shots);
			/* read shot gather */
			matrix_transpose(trans, dobs, nt, ng);
			<@\hyperref[matrix_transpose]{! GOTO matrix\_transpose( )}@>
			/* transpose the matrix to get dobs */
			if(csdgather){
				gxbeg=sxbeg+is*jsx-distx;
				sg_init(gxz, gzbeg, gxbeg, jgz, jgx, ng, nz);
				<@\hyperref[sg_init]{! GOTO sg\_init( )}@>
			}
			/* receiver position initialize */
			memset(sp0[0], 0, nz*nx*sizeof(float));
			memset(sp1[0], 0, nz*nx*sizeof(float));
			for(it=0; it<nt; it++){
				add_source(sp1, &wlt[it], &sxz[is], 1, nz, true);
				<@\hyperref[add_source]{! GOTO add\_source( )}@>
				/* add source */
								
				step_forward(sp0, sp1, sp2, vv, dtz, dtx, nz, nx);
				<@\hyperref[step_forward]{! GOTO step\_forward( )}@>
				/* forward exploration */
				
				ptr=sp0; sp0=sp1; sp1=sp2; sp2=ptr;
				/* update wavefield */
				
				record_seis(dcal, gxz, sp0, ng, nz);
				<@\hyperref[record_seis]{! GOTO record\_seis( )}@>
				/* save seismic record at recevier position */
				
				sum_alpha12(alpha1, alpha2, dcal, &dobs[it*ng], &derr[is*ng*nt+it*ng], ng);
				<@\label{sum_alpha12_in_main}
				\hyperref[sum_alpha12]{! GOTO sum\_alpha12( )}@>
				/* calculate alpha12 */
			}
		}
	
		alpha=cal_alpha(alpha1, alpha2, epsil, ng);
		<@\label{cal_alpha_in_main}
		\hyperref[cal_alpha]{! GOTO cal\_alpha( )}@>
		/* calculate alpha */
		
		update_vel(vv, cg, alpha, nz, nx);
		<@\label{update_vel_in_main}
		\hyperref[update_vel]{! GOTO update\_vel( )}@>
		/* update velocity */
		sf_floatwrite(vv[0], nz*nx, vupdates);
		/* output velcotiy */
		/* updating velocity complete */

		if(iter==0) {
			obj1=obj; 
			objval[iter]=1.0;
		} else{
			objval[iter]=obj/obj1;
		}
		/* calcuate obj */

		if(verb) {
			sf_warning("obj=%f  beta=%f  epsil=%f  alpha=%f", obj, beta, epsil, alpha);
			/* output important information at each FWI iteration */
			stop=clock();
			/* record ending time */ 
			sf_warning("iteration %d finished: %f (s)",iter+1, ((float)(stop-start))/CLOCKS_PER_SEC);
		}
	}
	sf_floatwrite(objval, niter, objs);
	/* output obj */
	
	free(*vv); free(vv);
	free(*vtmp); free(vtmp);
	free(*sp0); free(sp0);
	free(*sp1); free(sp1);
	free(*sp2); free(sp2);
	free(*gp0); free(gp0);
	free(*gp1); free(gp1);
	free(*gp2); free(gp2);
	free(*g0); free(g0);
	free(*g1); free(g1);
	free(*cg); free(cg);
	free(*lap); free(lap);
	free(*illum); free(illum);
	free(objval);
	free(wlt);
	free(sxz);
	free(gxz);
	free(bndr);
	free(trans);
	free(dobs);
	free(dcal);
	free(derr);
	free(alpha1);
	free(alpha2);

	exit(0);
}

\end{lstlisting}

\section*{sg\_init( )}
sg\_init( ) in \url{$(RSFROOT)/src/user/pyang/Mfwi2d.c}.
\begin{lstlisting}[label={sg_init},language=C,tabsize=4,caption=sg\_init( )]
	void sg_init(int *sxz, int szbeg, int sxbeg, 
	             int jsz, int jsx, int ns, int nz)
	/*< shot/geophone position initialize >*/
	{
		int is, sz, sx;
		for(is=0; is<ns; is++) {
			sz=szbeg+is*jsz;
			sx=sxbeg+is*jsx;
			sxz[is]=sz+nz*sx;
		}
	}
	<@\hyperref[sg_init_in_main]{//! RETURN main( )}@>
\end{lstlisting}

\section*{matrix\_transpose( )}
matrix\_transpose( ) in \url{$(RSFROOT)/src/user/pyang/Mfwi2d.c}.
\begin{lstlisting}[label={matrix_transpose},language=C,tabsize=4,caption=matrix\_transpose( )]
void matrix_transpose(float *matrix, float *trans, int n1, int n2)
/*< matrix transpose: matrix tansposed to be trans >*/
{
	int i1, i2;
	
	for(i2=0; i2<n2; i2++)
		for(i1=0; i1<n1; i1++)
			trans[i2+n2*i1]=matrix[i1+n1*i2];
}
<@\hyperref[matrix_transpose_in_main]{//! RETURN main( )}@>
\end{lstlisting}

\section*{add\_source( )}
add\_source( ) in \url{$(RSFROOT)/src/user/pyang/Mfwi2d.c}.
\begin{lstlisting}[label={add_source},language=C,tabsize=4,caption=add\_source( )]
void add_source(float **p, float *source, int *sxz, int ns, int nz, bool add)
/*< add/subtract seismic sources >*/
{
	int is, sx, sz;
	if(add){
		for(is=0;is<ns; is++){
			sx=sxz[is]/nz;
			sz=sxz[is]%nz;
			p[sx][sz]+=source[is];
		}
	}else{
		for(is=0;is<ns; is++){
			sx=sxz[is]/nz;
			sz=sxz[is]%nz;
			p[sx][sz]-=source[is];
		}
	}
}
<@\hyperref[add_source_in_main]{//! RETURN main( )}@>
\end{lstlisting}

\section*{step\_forward( )}
step\_forward( ) in \url{$(RSFROOT)/src/user/pyang/Mfwi2d.c}.
\begin{lstlisting}[label={step_forward},language=C,tabsize=4,caption=step\_forward( )]
void step_forward(float **p0, float **p1, float **p2, float **vv, float dtz, float dtx, int nz, int nx)
/*< forward modeling step, Clayton-Enquist ABC incorporated >*/
{
	int ix,iz;
	float v1,v2,diff1,diff2;
	
	for(ix=0; ix < nx; ix++){
		for(iz=0; iz < nz; iz++){
			v1=vv[ix][iz]*dtz; v1=v1*v1;
			v2=vv[ix][iz]*dtx; v2=v2*v2;
			diff1=diff2=-2.0*p1[ix][iz];
			diff1+=(iz-1>=0)?p1[ix][iz-1]:0.0;
			diff1+=(iz+1<nz)?p1[ix][iz+1]:0.0;
			diff2+=(ix-1>=0)?p1[ix-1][iz]:0.0;
			diff2+=(ix+1<nx)?p1[ix+1][iz]:0.0;
			diff1*=v1;
			diff2*=v2;
			p2[ix][iz]=2.0*p1[ix][iz]-p0[ix][iz]+diff1+diff2;
		}
	}
	for (ix=1; ix < nx-1; ix++) { 
		/* top boundary */
		/*
		iz=0;
		diff1=	(p1[ix][iz+1]-p1[ix][iz])-
		(p0[ix][iz+1]-p0[ix][iz]);
		diff2=	c21*(p1[ix-1][iz]+p1[ix+1][iz]) +
		c22*(p1[ix-2][iz]+p1[ix+2][iz]) +
		c20*p1[ix][iz];
		diff1*=sqrtf(vv[ix][iz])/dz;
		diff2*=vv[ix][iz]/(2.0*dx*dx);
		p2[ix][iz]=2*p1[ix][iz]-p0[ix][iz]+diff1+diff2;
		*/
		/* bottom boundary */
		iz=nz-1;
		v1=vv[ix][iz]*dtz; 
		v2=vv[ix][iz]*dtx;
		diff1=-(p1[ix][iz]-p1[ix][iz-1])+(p0[ix][iz]-p0[ix][iz-1]);
		diff2=p1[ix-1][iz]-2.0*p1[ix][iz]+p1[ix+1][iz];
		diff1*=v1;
		diff2*=0.5*v2*v2;
		p2[ix][iz]=2.0*p1[ix][iz]-p0[ix][iz]+diff1+diff2;
	}
	
	for (iz=1; iz <nz-1; iz++){ 
		/* left boundary */
		ix=0;
		v1=vv[ix][iz]*dtz; 
		v2=vv[ix][iz]*dtx;
		diff1=p1[ix][iz-1]-2.0*p1[ix][iz]+p1[ix][iz+1];
		diff2=(p1[ix+1][iz]-p1[ix][iz])-(p0[ix+1][iz]-p0[ix][iz]);
		diff1*=0.5*v1*v1;
		diff2*=v2;
		p2[ix][iz]=2.0*p1[ix][iz]-p0[ix][iz]+diff1+diff2;
		/* right boundary */
		ix=nx-1;
		v1=vv[ix][iz]*dtz; 
		v2=vv[ix][iz]*dtx;
		diff1=p1[ix][iz-1]-2.0*p1[ix][iz]+p1[ix][iz+1];
		diff2=-(p1[ix][iz]-p1[ix-1][iz])+(p0[ix][iz]-p0[ix-1][iz]);
		diff1*=0.5*v1*v1;
		diff2*=v2;
		p2[ix][iz]=2.0*p1[ix][iz]-p0[ix][iz]+diff1+diff2;
	}  
}
<@\hyperref[step_forward_in_main]{//! RETURN main( )}@>
\end{lstlisting}

\section*{rw\_bndr( )}
rw\_bndr( ) in \url{$(RSFROOT)/src/user/pyang/Mfwi2d.c}.
\begin{lstlisting}[label={rw_bndr},language=C,tabsize=4,caption=rw\_bndr( )]
void rw_bndr(float *bndr, float **p, int nz, int nx, bool write)
	/*< if write==true, write/save boundaries out of variables;
	else  read boundaries into variables (for 2nd order FD) >*/
{
	int i;
	if(write){
		for(i=0; i<nz; i++){
			bndr[i]=p[0][i];
			bndr[i+nz]=p[nx-1][i];
		}
		for(i=0; i<nx; i++) 
			bndr[i+2*nz]=p[i][nz-1];
	}else{
		for(i=0; i<nz; i++){
			p[0][i]=bndr[i];
			p[nx-1][i]=bndr[i+nz];
		}
		for(i=0; i<nx; i++) 
			p[i][nz-1]=bndr[i+2*nz];
		}
}
<@\hyperref[rw_bndr_in_main]{//! RETURN main( )}@>
\end{lstlisting}

\section*{record\_seis( )}
record\_seis( ) in \url{$(RSFROOT)/src/user/pyang/Mfwi2d.c}.
\begin{lstlisting}[label={record_seis},language=C,tabsize=4,caption=record\_seis( )]
void record_seis(float *seis_it, int *gxz, float **p, int ng, int nz)
/*< record seismogram at time it into a vector length of ng >*/
{
	int ig, gx, gz;
	for(ig=0;ig<ng; ig++){
		gx=gxz[ig]/nz;
		gz=gxz[ig]%nz;
		seis_it[ig]=p[gx][gz];
	}
}
<@\hyperref[record_seis_in_main]{//! RETURN main( )}@>
\end{lstlisting}

\section*{cal\_residuals( )}
cal\_residuals( ) in \url{$(RSFROOT)/src/user/pyang/Mfwi2d.c}.
\begin{lstlisting}[label={cal_residuals},language=C,tabsize=4,caption=cal\_residuals( )]
void cal_residuals(float *dcal, float *dobs, float *dres, int ng)
/*< calculate residual >*/
{
	int ig;
	for(ig=0; ig<ng; ig++){
		dres[ig]=dcal[ig]-dobs[ig];
	}
}
<@\hyperref[cal_residuals_in_main]{//! RETURN main( )}@>
\end{lstlisting}

\section*{step\_backward( )}
step\_backward( ) in \url{$(RSFROOT)/src/user/pyang/Mfwi2d.c}.
\begin{lstlisting}[label={step_backward},language=C,tabsize=4,caption=step\_backward( )]
void step_backward(float **illum, float **lap, float **p0, float **p1, float **p2, float **vv, float dtz, float dtx, int nz, int nx)
/*< step backward >*/
{
	int ix,iz;
	float v1,v2,diff1,diff2;    
	
	for(ix=0; ix < nx; ix++){
		for (iz=0; iz < nz; iz++){
			v1=vv[ix][iz]*dtz; v1=v1*v1;
			v2=vv[ix][iz]*dtx; v2=v2*v2;
			diff1=diff2=-2.0*p1[ix][iz];
			diff1+=(iz-1>=0)?p1[ix][iz-1]:0.0;
			diff1+=(iz+1<nz)?p1[ix][iz+1]:0.0;
			diff2+=(ix-1>=0)?p1[ix-1][iz]:0.0;
			diff2+=(ix+1<nx)?p1[ix+1][iz]:0.0;
			lap[ix][iz]=diff1+diff2;
			diff1*=v1;
			diff2*=v2;
			p2[ix][iz]=2.0*p1[ix][iz]-p0[ix][iz]+diff1+diff2;
			illum[ix][iz]+=p1[ix][iz]*p1[ix][iz];
		}
	}
}
<@\hyperref[step_backward_in_main]{//! RETURN main( )}@>
\end{lstlisting}

\section*{cal\_gradient( )}
cal\_gradient( ) in \url{$(RSFROOT)/src/user/pyang/Mfwi2d.c}.
\begin{lstlisting}[label={cal_gradient},language=C,tabsize=4,caption=cal\_gradient( )]
void cal_gradient(float **grad, float **lap, float **gp, int nz, int nx)
/*< calculate gradient >*/
{
	int ix, iz;
		for(ix=0; ix<nx; ix++){
			for(iz=0; iz<nz; iz++){
				grad[ix][iz]+=lap[ix][iz]*gp[ix][iz];
			}
		}
}
<@\hyperref[cal_gradient_in_main]{//! RETURN main( )}@>
\end{lstlisting}

\section*{cal\_objective( )}
cal\_objective( ) in \url{$(RSFROOT)/src/user/pyang/Mfwi2d.c}.
\begin{lstlisting}[label={cal_objective},language=C,tabsize=4,caption=cal\_objective( )]
float cal_objective(float *dres, int ng)
/*< calculate the value of objective function >*/
{
	int i;
	float a, obj=0;
	
	for(i=0; i<ng; i++){
		a=dres[i];
		obj+=a*a;
	}
	return obj;
}
<@\hyperref[cal_objective_in_main]{//! RETURN main( )}@>
\end{lstlisting}

\section*{scale\_gradient( )}
scale\_gradient( ) in \url{$(RSFROOT)/src/user/pyang/Mfwi2d.c}.
\begin{lstlisting}[label={scale_gradient},language=C,tabsize=4,caption=scale\_gradient( )]
void scale_gradient(float **grad, float **vv, float **illum, int nz, int nx, bool precon)
/*< scale gradient >*/
{
	int ix, iz;
	float a;
	for(ix=1; ix<nx-1; ix++){
		for(iz=1; iz<nz-1; iz++){
			a=vv[ix][iz];
			if (precon) 
				a*=sqrtf(illum[ix][iz]+SF_EPS);
				/*precondition with residual wavefield illumination*/
			grad[ix][iz]*=2.0/a;
		}
	}
	for(ix=0; ix<nx; ix++){
		grad[ix][0]=grad[ix][1];
		grad[ix][nz-1]=grad[ix][nz-2];
	}

	for(iz=0; iz<nz; iz++){
		grad[0][iz]=grad[1][iz];
		grad[nx-1][iz]=grad[nx-2][iz];
	}
}
<@\hyperref[scale_gradient_in_main]{//! RETURN main( )}@>
\end{lstlisting}

\section*{bell\_smoothz( )}
bell\_smoothz( ) in \url{$(RSFROOT)/src/user/pyang/Mfwi2d.c}.
\begin{lstlisting}[label={bell_smoothz},language=C,tabsize=4,caption=bell\_smoothz( )]
void bell_smoothz(float **g, float **smg, int rbell, int nz, int nx)
/*< gaussian bell smoothing for z-axis >*/
{
	int ix, iz, i;
	float s;
	
	for(ix=0; ix<nx; ix++){
		for(iz=0; iz<nz; iz++){
			s=0.0;
			for(i=-rbell; i<=rbell; i++)
			if(iz+i>=0 && iz+i<nz) 
			s+=expf(-(2.0*i*i)/rbell)*g[ix][iz+i];
			smg[ix][iz]=s;		
		}
	}
}
<@\hyperref[bell_smoothz_in_main]{//! RETURN main( )}@>
\end{lstlisting}

\section*{bell\_smoothx( )}
bell\_smoothx( ) in \url{$(RSFROOT)/src/user/pyang/Mfwi2d.c}.
\begin{lstlisting}[label={bell_smoothx},language=C,tabsize=4,caption=bell\_smoothx( )]
void bell_smoothx(float **g, float **smg, int rbell, int nz, int nx)
/*< gaussian bell smoothing for x-axis >*/
{
	int ix, iz, i;
	float s;
	
	for(ix=0; ix<nx; ix++) {
		for(iz=0; iz<nz; iz++){
			s=0.0;
			for(i=-rbell; i<=rbell; i++) 
				if(ix+i>=0 && ix+i<nx) 
					s+=expf(-(2.0*i*i)/rbell)*g[ix+i][iz];
			smg[ix][iz]=s;			
		}
	}
}
<@\hyperref[bell_smoothx_in_main]{//! RETURN main( )}@>
\end{lstlisting}

\section*{cal\_beta( )}
cal\_beta( ) in \url{$(RSFROOT)/src/user/pyang/Mfwi2d.c}.
\begin{lstlisting}[label={cal_beta},language=C,tabsize=4,caption=cal\_beta( )]
float cal_beta(float **g0, float **g1, float **cg, int nz, int nx)
/*< calculate beta >*/
{
	int ix, iz;
	float a,b,c;
	
	a=b=c=0;
	for(ix=0; ix<nx; ix++){
		for(iz=0; iz<nz; iz++){
			a += g1[ix][iz]*(g1[ix][iz]-g0[ix][iz]);// numerator of HS
			b += cg[ix][iz]*(g1[ix][iz]-g0[ix][iz]);// denominator of HS,DY
			c += g1[ix][iz]*g1[ix][iz];				// numerator of DY
		}
	}
	
	float beta_HS=(fabsf(b)>0)?(a/b):0.0; 
	float beta_DY=(fabsf(b)>0)?(c/b):0.0;
	return SF_MAX(0.0, SF_MIN(beta_HS, beta_DY));
}
<@\hyperref[cal_beta_in_main]{//! RETURN main( )}@>
\end{lstlisting}

\section*{cal\_conjgrad( )}
cal\_conjgrad( ) in \url{$(RSFROOT)/src/user/pyang/Mfwi2d.c}.
\begin{lstlisting}[label={cal_conjgrad},language=C,tabsize=4,caption=cal\_conjgrad( )]
void cal_conjgrad(float **g1, float **cg, float beta, int nz, int nx)
/*< calculate conjugate gradient >*/
{
	int ix, iz;

	for(ix=0; ix<nx; ix++){
		for(iz=0; iz<nz; iz++){
			cg[ix][iz]=-g1[ix][iz]+beta*cg[ix][iz];
		}
	}
}
<@\hyperref[cal_conjgrad_in_main]{//! RETURN main( )}@>
\end{lstlisting}

\section*{cal\_epsilon( )}
cal\_epsilon( ) in \url{$(RSFROOT)/src/user/pyang/Mfwi2d.c}.
\begin{lstlisting}[label={cal_epsilon},language=C,tabsize=4,caption=cal\_epsilon( )]
float cal_epsilon(float **vv, float **cg, int nz, int nx)
/*< calculate epsilcon >*/
{
	int ix, iz;
	float vvmax, cgmax;
	vvmax=cgmax=0.0;
	
	for(ix=0; ix<nx; ix++){
		for(iz=0; iz<nz; iz++){
			vvmax=SF_MAX(vvmax, fabsf(vv[ix][iz]));
			cgmax=SF_MAX(cgmax, fabsf(cg[ix][iz]));
		}
	}
	
	return 0.01*vvmax/(cgmax+SF_EPS);
}
<@\hyperref[cal_epsilon_in_main]{//! RETURN main( )}@>
\end{lstlisting}

\section*{cal\_vtmp( )}
cal\_vtmp( ) in \url{$(RSFROOT)/src/user/pyang/Mfwi2d.c}.
\begin{lstlisting}[label={cal_vtmp},language=C,tabsize=4,caption=cal\_vtmp( )]
void cal_vtmp(float **vtmp, float **vv, float **cg, float epsil, int nz, int nx)
/*< calculate temporary velcity >*/
{
	int ix, iz;

	for(ix=0; ix<nx; ix++){
		for(iz=0; iz<nz; iz++){
			vtmp[ix][iz]=vv[ix][iz]+epsil*cg[ix][iz];
		}
	}
}
<@\hyperref[cal_vtmp_in_main]{//! RETURN main( )}@>
\end{lstlisting}

\section*{sum\_alpha12( )}
sum\_alpha12( ) in \url{$(RSFROOT)/src/user/pyang/Mfwi2d.c}.
\begin{lstlisting}[label={sum_alpha12},language=C,tabsize=4,caption=sum\_alpha12( )]
void sum_alpha12(float *alpha1, float *alpha2, float *dcaltmp, float *dobs, float *derr, int ng)
/*< calculate numerator and denominator of alpha >*/
{
	int ig;
	float a, b, c;
	for(ig=0; ig<ng; ig++){
		c=derr[ig];
		a=dobs[ig]+c;	
		/* 
		 * since f(mk)-dobs[id]=derr[id], 
		 * thus f(mk)=b+c; 
		 */
		b=dcaltmp[ig]-a;
		/* f(mk+epsil*cg)-f(mk) */
		alpha1[ig]-=b*c; 
		alpha2[ig]+=b*b; 
	}
}
<@\hyperref[sum_alpha12_in_main]{//! RETURN main( )}@>
\end{lstlisting}

\section*{cal\_alpha( )}
cal\_alpha( ) in \url{$(RSFROOT)/src/user/pyang/Mfwi2d.c}.
\begin{lstlisting}[label={cal_alpha},language=C,tabsize=4,caption=cal\_alpha( )]
float cal_alpha(float *alpha1, float *alpha2, float epsil, int ng)
/*< calculate alpha >*/
{
	int ig;
	float a,b;

	a=b=0;
	for(ig=0; ig<ng; ig++){
		a+=alpha1[ig];
		b+=alpha2[ig];
	}

	return (a*epsil/(b+SF_EPS));
}
<@\hyperref[cal_alpha_in_main]{//! RETURN main( )}@>
\end{lstlisting}

\section*{update\_vel( )}
update\_vel( ) in \url{$(RSFROOT)/src/user/pyang/Mfwi2d.c}.
\begin{lstlisting}[label={update_vel},language=C,tabsize=4,caption=update\_vel( )]
void update_vel(float **vv, float **cg, float alpha, int nz, int nx)
/*< update velcity >*/
{
	int ix, iz;
	
	for(ix=0; ix<nx; ix++){
		for(iz=0; iz<nz; iz++){
			vv[ix][iz]+=alpha*cg[ix][iz];
		}
	}
}
<@\hyperref[update_vel_in_main]{//! RETURN main( )}@>
\end{lstlisting}

\end{document}
