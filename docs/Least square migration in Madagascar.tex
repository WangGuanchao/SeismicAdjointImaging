%%% Template originaly created by Karol Kozioł (mail@karol-koziol.net) and modified for ShareLaTeX use

\documentclass[a4paper,11pt]{article}

\usepackage[T1]{fontenc}
\usepackage[utf8]{inputenc}
\usepackage{graphicx}
\usepackage{xcolor}

\renewcommand\familydefault{\sfdefault}
\usepackage{tgheros}
\usepackage[defaultmono]{droidmono}

\usepackage{amsmath,amssymb,amsthm,textcomp}
\usepackage{enumerate}
\usepackage{multicol}
\usepackage{tikz}
\usepackage[colorlinks,linkcolor=blue]{hyperref}
\usepackage{geometry}
\geometry{total={210mm,297mm},
left=20mm,right=20mm,%
bindingoffset=0mm, top=20mm,bottom=20mm}

\linespread{1.3}

\newcommand{\linia}{\rule{\linewidth}{0.5pt}}

% custom theorems if needed
\newtheoremstyle{mytheor}
    {1ex}{1ex}{\normalfont}{0pt}{\scshape}{.}{1ex}
    {{\thmname{#1 }}{\thmnumber{#2}}{\thmnote{ (#3)}}}

\theoremstyle{mytheor}
\newtheorem{defi}{Definition}

% my own titles
\makeatletter
\renewcommand{\maketitle}{
\begin{center}
\vspace{2ex}
{\huge \textsc{\@title}}
\vspace{1ex}
\\
\linia\\
\@author \hfill \@date
\vspace{4ex}
\end{center}
}
\makeatother
%%%

% custom footers and headers
\usepackage{fancyhdr}
\pagestyle{fancy}
\lhead{}
\chead{}
\rhead{}
\cfoot{}
\rfoot{Page \thepage}
\renewcommand{\headrulewidth}{0pt}
\renewcommand{\footrulewidth}{0pt}
%

% code listing settings
\usepackage{listings}
\lstset{
    language=Python,
    basicstyle=\ttfamily\small,
    aboveskip={1.0\baselineskip},
    belowskip={1.0\baselineskip},
    columns=fixed,
    extendedchars=true,
    breaklines=true,
    tabsize=4,
    prebreak=\raisebox{0ex}[0ex][0ex]{\ensuremath{\hookleftarrow}},
    frame=lines,
    showtabs=false,
    showspaces=false,
    showstringspaces=false,
    keywordstyle=\color[rgb]{0.627,0.126,0.941},
    commentstyle=\color[rgb]{0.133,0.545,0.133},
    stringstyle=\color[rgb]{01,0,0},
    numbers=left,
    numberstyle=\small,
    stepnumber=1,
    numbersep=10pt,
    captionpos=t,
    escapeinside={<@}{@>}
}

%%%----------%%%----------%%%----------%%%----------%%%

\begin{document}

\title{Least Square Migration}

\author{Hou, Sian - sianhou1987@outlook.com}

\date{Jan/10/2017}

\maketitle

\section*{Introduction}
This is an explantion of Least Square Migration program in Madagascar (\url{https://github.com/ahay/src}) to help us understand the details of seismic inversion workflow. The author of code is Pengliang Yang and the theory can be found on \url{ http://www.reproducibility.org/RSF/book/xjtu/primer/paper_html/}. What's more, Karol Koziol published the {\LaTeX} template on ShareLatex \url{https://www.sharelatex.com/}.

Main points: 

1. You'd better to read the "Full Waveform Inversion in Madagascar.pdf" firstly.
\section*{main( )}
main( ) in \url{$(RSFROOT)/src/user/pyang/Mlsprtm2d.c}.
\begin{lstlisting}[label={main},language=C,tabsize=4,caption=main()]
/* 2-D prestack least-squares RTM using wavefield reconstruction
   NB: Sponge ABC is applied!
*/
/*
   Copyright (C) 2014  Xi'an Jiaotong University (Pengliang Yang)

   This program is free software; you can redistribute it and/or modify
   it under the terms of the GNU General Public License as published by
   the Free Software Foundation; either version 2 of the License, or
   (at your option) any later version.

   This program is distributed in the hope that it will be useful,
   but WITHOUT ANY WARRANTY; without even the implied warranty of
   MERCHANTABILITY or FITNESS FOR A PARTICULAR PURPOSE.  See the
   GNU General Public License for more details.

   You should have received a copy of the GNU General Public License
   along with this program; if not, write to the Free Software
   Foundation, Inc., 59 Temple Place, Suite 330, Boston, MA  02111-1307  USA
*/

#include <rsf.h>

#include "prtm2d.h"

int main(int argc, char* argv[]){   
	
	bool verb;
	int nb, nz, nx, nt, ns, ng, niter, csd, sxbeg, szbeg, jsx, jsz, gxbeg, gzbeg, jgx, jgz;
	float dz, dx, dt, fm, o1, o2, amp;
	float **v0, *mod, *dat;      
	
	//! I/O files
	sf_file shots, imag, imgrtm, velo;
	
	//! initialize Madagascar
	sf_init(argc,argv);
	
	shots = sf_input ("in");
	/* shot records, data */
	velo = sf_input ("vel"); 
	/* velocity */
	imag = sf_output("out"); 
	/* output LSRTM image, model */
	imgrtm = sf_output("imgrtm"); 
	/* output RTM image */
	
	if (!sf_histint(velo,"n1",&nz)) sf_error("n1");
	/* 1st dimension size */
	if (!sf_histint(velo,"n2",&nx)) sf_error("n2");
	/* 2nd dimension size */
	if (!sf_histfloat(velo,"d1",&dz)) sf_error("d1");
	/* d1 */
	if (!sf_histfloat(velo,"d2",&dx)) sf_error("d2");
	/* d2 */
	if (!sf_histfloat(velo,"o1",&o1)) sf_error("o1");
	/* o1 */
	if (!sf_histfloat(velo,"o2",&o2)) sf_error("o2");
	/* o2 */
	if (!sf_getbool("verb",&verb)) verb=true;
	/* verbosity */
	if (!sf_getint("niter",&niter)) niter=10;
	/* totol number of least-squares iteration*/
	if (!sf_getint("nb",&nb)) nb=20;
	/* number (thickness) of ABC on each side */   
	if (!sf_histint(shots,"n1",&nt)) sf_error("no nt");
	/* total modeling time steps */
	if (!sf_histint(shots,"n2",&ng)) sf_error("no ng");
	/* total receivers in each shot */
	if (!sf_histint(shots,"n3",&ns)) sf_error("no ns");
	/* number of shots */
	if (!sf_histfloat(shots,"d1",&dt)) sf_error("no dt");
	/* time sampling interval */
	if (!sf_histfloat(shots,"amp",&amp)) sf_error("no amp");
	/* maximum amplitude of ricker */
	if (!sf_histfloat(shots,"fm",&fm)) sf_error("no fm");
	/* dominant freq of ricker */
	if (!sf_histint(shots,"sxbeg",&sxbeg)) sf_error("no sxbeg");
	/* x-begining index of sources, starting from 0 */
	if (!sf_histint(shots,"szbeg",&szbeg)) sf_error("no szbeg");
	/* x-begining index of sources, starting from 0 */
	if (!sf_histint(shots,"gxbeg",&gxbeg)) sf_error("no gxbeg");
	/* x-begining index of receivers, starting from 0 */
	if (!sf_histint(shots,"gzbeg",&gzbeg)) sf_error("no gzbeg");
	/* x-begining index of receivers, starting from 0 */
	if (!sf_histint(shots,"jsx",&jsx)) sf_error("no jsx");
	/* source x-axis  jump interval */
	if (!sf_histint(shots,"jsz",&jsz)) sf_error("no jsz");
	/* source z-axis jump interval */
	if (!sf_histint(shots,"jgx",&jgx)) sf_error("no jgx");
	/* receiver x-axis jump interval */
	if (!sf_histint(shots,"jgz",&jgz)) sf_error("no jgz");
	/* receiver z-axis jump interval  */
	if (!sf_histint(shots,"csdgather",&csd)) 
		sf_error("csdgather or not required");
	/* default, common shot-gather; if n, record at every point*/
	
	sf_putint(imag,"n1",nz);
	sf_putint(imag,"n2",nx);
	sf_putint(imag,"n3",1);
	sf_putfloat(imag,"d1",dz);
	sf_putfloat(imag,"d2",dx);
	sf_putfloat(imag,"o1",o1);
	sf_putfloat(imag,"o2",o2);
	sf_putstring(imag,"label1","Depth");
	sf_putstring(imag,"label2","Distance");
	/* output LSRTM image, model */
	
	sf_putint(imgrtm,"n1",nz);
	sf_putint(imgrtm,"n2",nx);
	sf_putint(imgrtm,"n3",1);
	sf_putfloat(imgrtm,"d1",dz);
	sf_putfloat(imgrtm,"d2",dx);
	sf_putfloat(imgrtm,"o1",o1);
	sf_putfloat(imgrtm,"o2",o2);
	sf_putstring(imgrtm,"label1","Depth");
	sf_putstring(imgrtm,"label2","Distance");
	/* output RTM image */

	v0=sf_floatalloc2(nz,nx);
	mod=sf_floatalloc(nz*nx);
	dat=sf_floatalloc(nt*ng*ns);
	/* 
	 * In rtm, vv is the velocity model [modl], which is input parameter; 
	 * mod is the image/reflectivity [imag]; 
	 * dat is seismogram [data]! 
	 */
	
	//! initialize velocity, model and data
	sf_floatread(v0[0], nz*nx, velo);
	memset(mod, 0, nz*nx*sizeof(float));
	sf_floatread(dat, nt*ng*ns, shots);
	prtm2d_init(verb, csd, dz, dx, dt, amp, fm, nz, nx, nb, nt, ns, ng, 
	sxbeg, szbeg, jsx, jsz, gxbeg, gzbeg, jgx, jgz, v0, mod, dat);
	<@\label{prtm2d_init_in_main} \hyperref[prtm2d_init]{! GOTO prtm2d\_init( )}@>
	
	prtm2d_lop(true, false, nz*nx, nt*ng*ns, mod, dat); 
	<@\label{prtm2d_lop_in_main} \hyperref[prtm2d_lop]{! GOTO prtm2d\_lop( )}@>
	/* original RTM is simply applying adjoint of prtm2d_lop once!*/
	sf_floatwrite(mod, nz*nx, imgrtm);
	/* output RTM image */

	//! least squares migration
	sf_solver(prtm2d_lop, sf_cgstep, nz*nx, nt*ng*ns, mod, dat, niter, "verb", verb, "end");
	sf_floatwrite(mod, nz*nx, imag);  
	/* output inverted image */
	
	sf_cgstep_close();
	prtm2d_close();
	free(*v0); free(v0);
	free(mod);
	free(dat); 
	
	exit(0);
}
\end{lstlisting}

\section*{prtm2d.c}
prtm2d.c in \url{$(RSFROOT)/src/user/pyang/prtm2d.c}.
\begin{lstlisting}[label={prtm2d},language=C,tabsize=4,caption=prtm2d.c]
/* 2-D prestack LSRTM linear operator using wavefield reconstruction method
   Note: Sponge ABC is applied!
*/
/*
   Copyright (C) 2014  Xi'an Jiaotong University, UT Austin (Pengliang Yang)

   This program is free software; you can redistribute it and/or modify
   it under the terms of the GNU General Public License as published by
   the Free Software Foundation; either version 2 of the License, or
   (at your option) any later version.

   This program is distributed in the hope that it will be useful,
   but WITHOUT ANY WARRANTY; without even the implied warranty of
   MERCHANTABILITY or FITNESS FOR A PARTICULAR PURPOSE.  See the
   GNU General Public License for more details.

   You should have received a copy of the GNU General Public License
   along with this program; if not, write to the Free Software
   Foundation, Inc., 59 Temple Place, Suite 330, Boston, MA  02111-1307  USA
*/
#include <rsf.h>

#ifdef _OPENMP
#include <omp.h>
#endif

#include "prtm2d.h"

static bool csdgather, verb;
static int nzpad, nxpad, nb, nz, nx, nt, ns, ng, sxbeg, szbeg, jsx, jsz, gxbeg, gzbeg, jgx, jgz, distx, distz;
static int *sxz, *gxz;
static float c0, c11, c21, c12, c22;
static float *wlt, *bndr,*rwbndr, *mod, *dat;
static float **sp0, **sp1, **gp0, **gp1, **vv, **ptr=NULL;


void boundary_rw(float **p, float *spo, bool read)
<@\label{boundary_rw} \hyperref[boundary_rw_first]{! GO BACK}@>
/* read/write using effective boundary saving strategy: 
if read=true, read the boundary out; else save/write the boundary */
{
	int ix,iz;

	if (read){
		#ifdef _OPENMP
			#pragma omp parallel for			\
			private(ix,iz)				\
			shared(p,spo,nx,nz,nb)  
		#endif	
		for(ix=0; ix<nx; ix++){
			for(iz=0; iz<2; iz++){
				p[ix+nb][iz-2+nb]=spo[iz+4*ix];
				p[ix+nb][iz+nz+nb]=spo[iz+2+4*ix];
			}
		}
		#ifdef _OPENMP
			#pragma omp parallel for			\
			private(ix,iz)				\
			shared(p,spo,nx,nz,nb)  
		#endif	
		for(iz=0; iz<nz; iz++){
			for(ix=0; ix<2; ix++){
				p[ix-2+nb][iz+nb]=spo[4*nx+iz+nz*ix];
				p[ix+nx+nb][iz+nb]=spo[4*nx+iz+nz*(ix+2)];
			}
		}
	} else {
		#ifdef _OPENMP
			#pragma omp parallel for			\
			private(ix,iz)				\
			shared(p,spo,nx,nz,nb)  
		#endif	
		for(ix=0; ix<nx; ix++){
			for(iz=0; iz<2; iz++){
				spo[iz+4*ix]=p[ix+nb][iz-2+nb];
				spo[iz+2+4*ix]=p[ix+nb][iz+nz+nb];
			}
		}
		#ifdef _OPENMP
			#pragma omp parallel for			\
			private(ix,iz)				\
			shared(p,spo,nx,nz,nb)  
		#endif	
		for(iz=0; iz<nz; iz++){
			for(ix=0; ix<2; ix++){
				spo[4*nx+iz+nz*ix]=p[ix-2+nb][iz+nb];
				spo[4*nx+iz+nz*(ix+2)]=p[ix+nx+nb][iz+nb];
			}
		}
	}
}

void step_forward(float **u0, float **u1, float **vv, bool adj)
<@\label{step_forward} \hyperref[step_forward_first]{! GO BACK}@>
/*< forward step for wave propagation >*/
{
	int i1, i2;

	if(adj){
		#ifdef _OPENMP
			#pragma omp parallel for default(none)			\
			private(i2,i1)					\
			shared(nzpad,nxpad,u1,vv,u0,c0,c11,c12,c21,c22)
		#endif
		for (i2=2; i2<nxpad-2; i2++) {
			for (i1=2; i1<nzpad-2; i1++) {
				u0[i2][i1]=2.*u1[i2][i1]-u0[i2][i1]+
				    c0*vv[i2][i1]*u1[i2][i1]+
					c11*(vv[i2][i1-1]*u1[i2][i1-1]+vv[i2][i1+1]*u1[i2][i1+1])+
					c12*(vv[i2][i1-2]*u1[i2][i1-2]+vv[i2][i1+2]*u1[i2][i1+2])+
					c21*(vv[i2-1][i1]*u1[i2-1][i1]+vv[i2+1][i1]*u1[i2+1][i1])+
					c22*(vv[i2-2][i1]*u1[i2-2][i1]+vv[i2+2][i1]*u1[i2+2][i1]);
			}
		}
	}else{
		#ifdef _OPENMP
			#pragma omp parallel for default(none)			\
			private(i2,i1)					\
			shared(nzpad,nxpad,u1,vv,u0,c0,c11,c12,c21,c22)
		#endif
		for (i2=2; i2<nxpad-2; i2++) {
			for (i1=2; i1<nzpad-2; i1++) {
				u0[i2][i1]=2.*u1[i2][i1]-u0[i2][i1]+ 
					vv[i2][i1]*(c0*u1[i2][i1]+
					c11*(u1[i2][i1-1]+u1[i2][i1+1])+
					c12*(u1[i2][i1-2]+u1[i2][i1+2])+
					c21*(u1[i2-1][i1]+u1[i2+1][i1])+
					c22*(u1[i2-2][i1]+u1[i2+2][i1]));
			}
		}
	}
}

void apply_sponge(float **p0)
/*< apply sponge (Gaussian taper) absorbing boundary condition
L=Gaussian taper ABC; L=L*, L is self-adjoint operator. >*/
{
	int ix,iz,ib,ibx,ibz;
	float w;

	for(ib=0; ib<nb; ib++) {
		w = bndr[ib];
		ibz = nzpad-ib-1;
		for(ix=0; ix<nxpad; ix++) {
			p0[ix][ib ] *= w; /*    top sponge */
			p0[ix][ibz] *= w; /* bottom sponge */
		}

		ibx = nxpad-ib-1;
		for(iz=0; iz<nzpad; iz++) {
			p0[ib ][iz] *= w; /*   left sponge */
			p0[ibx][iz] *= w; /*  right sponge */
		}
	}
}

void sg_init(int *sxz, int szbeg, int sxbeg, int jsz, int jsx, int ns)
/*< shot/geophone position initialize
	sxz/gxz; szbeg/gzbeg; 
	sxbeg/gxbeg; 
	jsz/jgz; jsx/jgx; ns/ng; >*/
{
	int is, sz, sx;

	for(is=0; is<ns; is++){
		sz=szbeg+is*jsz;
		sx=sxbeg+is*jsx;
		sxz[is]=sz+nz*sx;
	}
}

void add_source(int *sxz, float **p, int ns, float *source, bool add)
/*< add seismic sources in grid >*/
{
	int is, sx, sz;

	if(add){
	/* add sources*/
		#ifdef _OPENMP
			#pragma omp parallel for default(none)		\
			private(is,sx,sz)				\
			shared(p,source,sxz,nb,ns,nz)
		#endif
		for(is=0;is<ns; is++){
			sx=sxz[is]/nz;
			sz=sxz[is]%nz;
			p[sx+nb][sz+nb]+=source[is];
		}
	}else{
	/* subtract sources */
		#ifdef _OPENMP
			#pragma omp parallel for default(none)		\
			private(is,sx,sz)				\
			shared(p,source,sxz,nb,ns,nz)
		#endif
		for(is=0;is<ns; is++){
			sx=sxz[is]/nz;
			sz=sxz[is]%nz;
			p[sx+nb][sz+nb]-=source[is];
		}
	}
}

void expand2d(float** b, float** a)
/*< expand domain of 'a' to 'b': source(a)-->destination(b) >*/
{
int iz,ix;

#ifdef _OPENMP
#pragma omp parallel for default(none)		\
private(ix,iz)				\
shared(b,a,nb,nz,nx)
#endif
for     (ix=0;ix<nx;ix++) {
for (iz=0;iz<nz;iz++) {
b[nb+ix][nb+iz] = a[ix][iz];
}
}

for     (ix=0; ix<nxpad; ix++) {
for (iz=0; iz<nb;    iz++) {
b[ix][      iz  ] = b[ix][nb  ];
b[ix][nzpad-iz-1] = b[ix][nzpad-nb-1];
}
}

for     (ix=0; ix<nb;    ix++) {
for (iz=0; iz<nzpad; iz++) {
b[ix 	 ][iz] = b[nb  		][iz];
b[nxpad-ix-1 ][iz] = b[nxpad-nb-1	][iz];
}
}
}


void window2d(float **a, float **b)
/*< window 'b' to 'a': source(b)-->destination(a) >*/
{
int iz,ix;

#ifdef _OPENMP
#pragma omp parallel for default(none)		\
private(ix,iz)				\
shared(b,a,nb,nz,nx)
#endif
for     (ix=0;ix<nx;ix++) {
for (iz=0;iz<nz;iz++) {
a[ix][iz]=b[nb+ix][nb+iz] ;
}
}
}


void prtm2d_init(bool verb_, bool csdgather_, 
                 float dz_, float dx_, float dt_, 
				 float amp, float fm, 
				 int nz_, int nx_, int nb_, int nt_, int ns_, int ng_, 
				 int sxbeg_, int szbeg_, int jsx_, int jsz_, 
				 int gxbeg_, int gzbeg_, int jgx_, int jgz_,
				 float **v0, float *mod_, float *dat_)
<@\label{prtm2d_init} \hyperref[prtm2d_init_in_main]{! GOTO main( )}@>
/*< allocate variables and initialize parameters >*/
{  
	#ifdef _OPENMP
		omp_init();
	#endif
	/* initialize OpenMP support */ 
	
	float t;
	int i1, i2, it,ib;
	t = 1.0/(dz_*dz_);
	c11 = 4.0*t/3.0;
	c12= -t/12.0;
	t = 1.0/(dx_*dx_);
	c21 = 4.0*t/3.0;
	c22= -t/12.0;
	c0=-2.0*(c11+c12+c21+c22);
	/* finite difference */

	verb=verb_;
	csdgather=csdgather_;
	ns=ns_;
	ng=ng_;
	nb=nb_;
	nz=nz_;
	nx=nx_;
	nt=nt_;
	sxbeg=sxbeg_;
	szbeg=szbeg_;
	jsx=jsx_;
	jsz=jsz_;
	gxbeg=gxbeg_;
	gzbeg=gzbeg_;
	jgx=jgx_;
	jgz=jgz_;

	nzpad=nz+2*nb;
	nxpad=nx+2*nb;

	//! allocate temporary arrays
	bndr=sf_floatalloc(nb);
	sp0=sf_floatalloc2(nzpad,nxpad);
	sp1=sf_floatalloc2(nzpad,nxpad);
	gp0=sf_floatalloc2(nzpad,nxpad);
	gp1=sf_floatalloc2(nzpad,nxpad);
	vv=sf_floatalloc2(nzpad,nxpad);
	wlt=sf_floatalloc(nt);
	sxz=sf_intalloc(ns);
	gxz=sf_intalloc(ng);
	rwbndr=sf_floatalloc(nt*4*(nx+nz));

	//! initialized sponge ABC coefficients
	for(ib=0;ib<nb;ib++){
		t=0.015*(nb-1-ib);
		bndr[ib]=expf(-t*t);
	}
	mod=mod_;
	dat=dat_;
	//! initialize model
	for (i2=0; i2<nx; i2++){
		for (i1=0; i1<nz; i1++){
			t=v0[i2][i1]*dt_;
			vv[i2+nb][i1+nb] = t*t;
		}	
	}
	for (i2=0; i2<nxpad; i2++){
		for (i1=0; i1<nb; i1++){
			vv[i2][   i1  ] =vv[i2][   nb  ];
			vv[i2][nzpad-i1-1] =vv[i2][nzpad-nb-1];
		}
	}
	for (i2=0; i2<nb; i2++){
		for (i1=0; i1<nzpad; i1++){
			vv[   i2  ][i1] =vv[   nb  ][i1];
			vv[nxpad-i2-1][i1] =vv[nxpad-nb-1][i1];
		}
	}
	//! initialize source
	for(it=0; it<nt; it++){
		t=SF_PI*fm*(it*dt_-1.0/fm);t=t*t;
		wlt[it]=amp*(1.0-2.0*t)*expf(-t);
	}

	//! configuration of sources and geophones
	if (!(sxbeg>=0 && szbeg>=0 && 
	      sxbeg+(ns-1)*jsx<nx && szbeg+(ns-1)*jsz<nz)) { 
		      sf_warning("sources exceeds the computing zone!"); 
		      exit(1);
	}
	sg_init(sxz, szbeg, sxbeg, jsz, jsx, ns);
	distx=sxbeg-gxbeg;
	distz=szbeg-gzbeg;
	if (!(gxbeg>=0 && gzbeg>=0 &&
	      gxbeg+(ng-1)*jgx<nx && gzbeg+(ng-1)*jgz<nz)) { 
		      sf_warning("geophones exceeds the computing zone!");
		      exit(1);
	}
	if (csdgather && 
		!((sxbeg+(ns-1)*jsx)+(ng-1)*jgx-distx <nx && (szbeg+(ns-1)*jsz)+(ng-1)*jgz-distz <nz)){
			sf_warning("geophones exceeds the computing zone!");
			exit(1);
	}
	sg_init(gxz, gzbeg, gxbeg, jgz, jgx, ng);
}

void prtm2d_close()
/*< free allocated variables >*/
{
free(bndr);
free(*sp0); free(sp0);
free(*sp1); free(sp1);
free(*gp0); free(gp0);
free(*gp1); free(gp1);
free(*vv); free(vv);
free(wlt);
free(sxz);
free(gxz);
}

void prtm2d_lop(bool adj, bool add, int nm, int nd, float *mod, float *dat)
<@\label{prtm2d_lop} \hyperref[prtm2d_lop_in_main]{! GOTO main( )}@>
/*< prtm2d linear operator >*/
{
	int i1,i2,it,is,ig, gx, gz;
	if(nm!=nx*nz) sf_error("model size mismatch: %d!=%d",nm, nx*nz);
	if(nd!=nt*ng*ns) sf_error("data size mismatch: %d!=%d",nd,nt*ng*ns);
	sf_adjnull(adj, add, nm, nd, mod, dat); 
	<@\label{sf_adjnull_first} \hyperref[sf_adjnull]{! GOTO sf\_adjnull( )}@>
	/* set mod = 0.0f */
	
	for(is=0; is<ns; is++) {
		//! initialize is-th source wavefield Ps[]
		memset(sp0[0], 0, nzpad*nxpad*sizeof(float));
		memset(sp1[0], 0, nzpad*nxpad*sizeof(float));
		memset(gp0[0], 0, nzpad*nxpad*sizeof(float));
		memset(gp1[0], 0, nzpad*nxpad*sizeof(float));
		
		if(csdgather){
			gxbeg=sxbeg+is*jsx-distx;
			sg_init(gxz, gzbeg, gxbeg, jgz, jgx, ng);
		}

		if(adj){
			//! migration: mm=Lt dd
			for(it=0; it<nt; it++){			
				add_source(&sxz[is], sp1, 1, &wlt[it], true);
				step_forward(sp0, sp1, vv, false);
				<@\label{step_forward_first} \hyperref[step_forward]{! GOTO step\_forward( )}@>
				apply_sponge(sp0);
				apply_sponge(sp1);
				ptr=sp0; sp0=sp1; sp1=ptr;
				boundary_rw(sp0, &rwbndr[it*4*(nx+nz)], false);
				<@\label{boundary_rw_first} \hyperref[boundary_rw]{! GOTO boundary\_rw( )}@>
			}

			for (it=nt-1; it >-1; it--) {
			/* reverse time order, Img[]+=Ps[]* Pg[]; */
				if(verb) sf_warning("%d;",it);

				//! reconstruct source wavefield Ps[]	
				boundary_rw(sp0, &rwbndr[it*4*(nx+nz)], true);
				<@\hyperref[boundary_rw]{! GOTO boundary\_rw( )}@>
				ptr=sp0; sp0=sp1; sp1=ptr;
				step_forward(sp0, sp1, vv, false);
				add_source(&sxz[is], sp1, 1, &wlt[it], false);

				//! backpropagate receiver wavefield
				for(ig=0;ig<ng; ig++){
					gx=gxz[ig]/nz;
					gz=gxz[ig]%nz;
					gp1[gx+nb][gz+nb]+=dat[it+ig*nt+is*nt*ng];
				}
				step_forward(gp0, gp1, vv, false);
				apply_sponge(gp0); 
				apply_sponge(gp1); 
				ptr=gp0; gp0=gp1; gp1=ptr;

				//! rtm imaging condition
				for(i2=0; i2<nx; i2++)
					for(i1=0; i1<nz; i1++)
						mod[i1+nz*i2]+=sp0[i2+nb][i1+nb]*gp1[i2+nb][i1+nb];
			}
	} else {
		//! Born modeling/demigration: dd=L mm
		for(it=0; it<nt; it++){	
		/* forward time order, Pg[]+=Ps[]* Img[]; */	
		if(verb) sf_warning("%d;",it);	

		for(i2=0; i2<nx; i2++)
			for(i1=0; i1<nz; i1++)
				gp1[i2+nb][i1+nb]+=sp0[i2+nb][i1+nb]*mod[i1+nz*i2];
				/* Born source */
				ptr=gp0; gp0=gp1; gp1=ptr;
				apply_sponge(gp0); 
				apply_sponge(gp1); 
				step_forward(gp0, gp1, vv, true);

				for(ig=0;ig<ng; ig++){
					gx=gxz[ig]/nz;
					gz=gxz[ig]%nz;
					dat[it+ig*nt+is*nt*ng]+=gp1[gx+nb][gz+nb];
				}

				add_source(&sxz[is], sp1, 1, &wlt[it], true);
				step_forward(sp0, sp1, vv, false);
				apply_sponge(sp0);
				apply_sponge(sp1);
				ptr=sp0; sp0=sp1; sp1=ptr;
			}	
		}
	}	 
}

void sf_adjnull(bool adj /* adjoint flag */, 
				bool add /* addition flag */, 
				int nx   /* size of x */, 
				int ny   /* size of y */, 
				float* x, 
				float* y)
<@\label{sf_adjnull} \hyperref[sf_adjnull_first]{! GO BACK}, \url{$(RSFROOT)/src/build/api/c/adjnull.c}@>
/*< Zeros out the output (unless add is true). 
Useful first step for any linear operator. >*/
{
	int i;

	if(add) return;

	if(adj) {
		for (i = 0; i < nx; i++) {
			x[i] = 0.;
		}
	} else {
		for (i = 0; i < ny; i++) {
			y[i] = 0.;
		}
	}
}
\end{lstlisting}

\end{document}
